%% question-11.tex
%%

%% ==============================
\subsection{\textsc{Ocl} - Contrainte d'unicité}
\label{sec:question11}
%% ==============================

La contrainte suivante (Listing \ref{lst:uniproc}) permet de vérifier que le nom d'une procédure est unique.

\begin{lstlisting}[caption={Unicité des noms des procédures},captionpos=b,label={lst:uniproc},language=OCL]
context Procedure inv:
	Procedure.allInstance()->forAll(p1, p2 
		| p1 <> p2 implies p1.name <> p2.name)
\end{lstlisting}

Afin de vérifier que la procédure \textbf{Cody} est bien unique, nous utilisons la contrainte OCL ci-dessous (Listing \ref{lst:unicody}).

\begin{lstlisting}[caption={Unicité de la procédure Cody dans le programme},captionpos=b,label={lst:unicody},language=OCL]
context Programme inv:
	self.declarations.size(declaration |
		declaration.oclIsTypeOf(Procedure) and 
		declaration.name == "Cody") == 1
\end{lstlisting}

La contrainte permettant de vérifier qu'une variable globale porte un nom unique est défini comme suit (Listing \ref{lst:univar}) :

\begin{lstlisting}[caption={Unicité des noms des variables},captionpos=b,label={lst:univar},language=OCL]
context Programme inv:
	self.declarations.forAll(d1, d2 |
		d1.oclIsTypeOf(Variable) and
		d2.oclIsTypeOf(Variable) and
		d1 <> d2 implies d1.name <> d2.name)
\end{lstlisting}

La contrainte ci-après (Listing \ref{lst:uniparam}) permet l'unicité des noms des paramètres d'une procédure.

\begin{lstlisting}[caption={Unicité des noms des paramètres d'une procédure},captionpos=b,label={lst:uniparam},language=OCL]
context Procedure inv:
	self.parametres.forAll(p1, p2 |
		p1 <> p2 implies p1.name <> p2.name)
\end{lstlisting}

Une procédure va aussi devoir avoir une unicité pour les noms des variables de celle-ci. Cette unicité est indiqué via la contrainte OCL suivante (Listing \ref{lst:univarproc})

\begin{lstlisting}[caption={Unicité des noms des variables d'une procédure},captionpos=b,label={lst:univarproc},language=OCL]
context Procedure inv:
	self.declarations.forAll(d1, d2 |
		d1.oclIsTypeOf(Variable) and
		d2.oclIsTypeOf(Variable) and
		d1 <> d2 implies d1.name <> d2.name)
\end{lstlisting}

Enfin, la contrainte OCL ci-dessous (Listing \ref{lst:unirecord}) permet l'unicité des noms des champs d'un Enregistrement

\begin{lstlisting}[caption={Unicité des noms des champs d'un enregistrement},captionpos=b,label={lst:unirecord},language=OCL]
context Enregistrement inv:
	self.champs.forAll(c1, c2 | 
		c1 <> c2 implies c1.name <> c2.name)
\end{lstlisting} 
