%% question-13.tex
%%

%% ==============================
\subsection{\textsc{Ocl} - Contrats sur l'opération \emph{type(exp : Expression) : Type}}
\label{sec:question13}
%% ==============================

Dans les différents contrats, la fonction \emph{oclType()} est utilisée. Il s'agit d'une fonction fourni par OCL qui permet d'évaluer le type de l'instance sur laquelle il est appelé \citep{Omg2012}.

Le contrat OCL présenté ci-dessous (Listing \ref{lst:type_type}) spécifie que le type des littéraux est le type qui leur correspond.

\begin{lstlisting}[caption=Le type des littéraux est le type qui leur correspond,captionpos=b,label={lst:type_type},language=OCL]
context Expression::type(exp: Expression)
	post:
		if exp.oclIsTypeOf(Primitif) then
			return exp.oclType()
\end{lstlisting}

Le contrat du listing \ref{lst:type_unaire} indique que le type d'une expression unaire est lié au type de l'opérateur, à condition que sa sous-expression y corresponde.

\begin{lstlisting}[caption=Contrat OCL sur le type unaire,captionpos=b,label={lst:type_unaire},language=OCL]
context Expression::type(exp: Expression)
	post:
		if exp.oclIsTypeOf(Unaire) then
			if exp.operateur.oclIsTypeOf(
				exp.sous-expression.oclType()
			) then
				return exp.operateur.oclType()
\end{lstlisting}

Le contrat OCL suivant (Listing \ref{lst:type_binaire}) défini qu'un type d'une expression binaire est lié au type de son opérateur.

\begin{lstlisting}[caption=Contrat OCL sur le type binaire,captionpos=b,label={lst:type_binaire},language=OCL]
context Expression::type(exp: Expression)
	post:
		if exp.oclIsTypeOf(Binaire) then
			if exp.operateur.oclIsTypeOf(
				exp.leftHandSide.oclType()
			) and exp.operateur.oclIsTypeOf(
				exp.rightHandSide.oclType()
			) then
				return exp.operateur.oclType()
\end{lstlisting}

Le contrat ci-dessous (Listing \ref{lst:type_paranthese}) spécifie que le type d'une expression paranthésée est le type de sa sous-expression.

\begin{lstlisting}[caption=Contrat OCL sur le type parenthèsée,captionpos=b,label={lst:type_paranthese},language=OCL]
context Expression::type(exp: Expression)
	post:
		if exp.oclIsTypeOf(Parenthese) then
			return exp.sous-expression.oclType()
\end{lstlisting}

Le contrat suivant (Listing \ref{lst:type_record}) spécifie que le type d'une expression gauche correspondant à l'accès à un champ est le type de sa déclaration dans l'enregistrement.

\begin{lstlisting}[caption=Contrat OCL sur le type d'un enregistrement,captionpos=b,label={lst:type_record},language=OCL]
context Expression::type(exp: Expression)
	post:
		if exp.oclIsTypeOf(Binaire) then
			if exp.rightHandSide.oclIsTypeOf(InvocationChamp) then
				return exp.rightHandSide.parent
					.getChamp(exp.rightHandSide.nomChamp)
					.oclType()
\end{lstlisting}

Le dernier contrat présenté dans cette section (Listing \ref{lst:type_array}) spécifie que le type d'une expression gauche correspondant à l'accès à une case d'un tableau est le type de la déclaration du tableau.

\begin{lstlisting}[caption=Contrat OCL sur le type d'un tableau,captionpos=b,label={lst:type_array},language=OCL]
context Expression::type(exp: Expression)
	post:
		if exp.oclIsTypeOf(Binaire) then
			if exp.rightHandSide.oclIsTypeOf(InvocationCellule) then
				return exp.rightHandSide.source.type.oclType()
\end{lstlisting}
