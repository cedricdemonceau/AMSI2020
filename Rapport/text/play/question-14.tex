%% question-14.tex
%%

%% ==============================
\subsection{\textsc{Ocl} - Contrats sur l'opération \emph{estValide() : Boolean}}
\label{sec:question14}
%% ==============================

Ce premier contrat OCL (Listing \ref{lst:valid_declaration}) spécifie qu'un appel de procédure doit référer à une déclaration de procédure dont le nom existe dans le programme.

\begin{lstlisting}[caption=Contrat sur l'existance du nom de la procédure appellée,captionpos=b,label={lst:valid_declaration},language=OCL]
context Declaration::estValide()
	post:
		if self.oclIsTypeOf(AppelProcedure) then
			return Procedure.allInstance()
				->include(p1 | p1.name == self.appel.name)
\end{lstlisting}

Les gardes des instructions sont une propriété boolean de la classe \emph{Composée} (Figure \ref{fig:instruction}), il n'y a donc pas besoin de contrat OCL pour vérifier que les gardes soient toujours des boolean et que l'instruction soit valide.

Le contrat OCL ci-dessous (Listing \ref{lst:valid_action}) spécifie que tous les paramètres d'une instruction \emph{Action} doivent être des entiers.

\begin{lstlisting}[caption=Les paramètres d'une instruction \emph{Action} doivent être des entiers,captionpos=b,label={lst:valid_action},language=OCL]
context Declaration::estValide()
	post:
		if self.oclIsTypeOf(ProcedureAction) then
			return self.parametres.forAll(
				p1 | p1.type.oclIsTypeOf(Entier)
			)
\end{lstlisting}


Le contrat OCL présenté à la figure \ref{lst:valid_param} spécifie que les paramètres effectifs de rang \emph{i} doivent être de même type dans un appel de procédure que dans la déclaration du même nom.

\begin{lstlisting}[caption=Contrat OCL indiquant que les paramètres effectifs doivent être de même type que dans la déclaration,captionpos=b,label={lst:valid_param},language=OCL]
context Declaration::estValide()
	if self.oclIsTypeOf(AppelProcedure) then
		return self.list.forAll(p1 |
			self.parametres.exists(p2 |
				p1.oclType() == p2.oclType() and
				p1.name == p2.name
			)
		)	
\end{lstlisting}

Le contrat suivant (Listing \ref{lst:valid_affectation}) indique que la partie gauche et la partie droite d'une affectation doivent être de même type.

\begin{lstlisting}[caption=Contrat OCL spécifiant que la partie gauche et droite d'une affectation ont le même type,captionpos=b,label={lst:valid_affectation},language=OCL]
context Declaration::estValide()
	post:
		if self.oclIsTypeOf(Affectation) then
			return Expression::type(self.leftHandSide) == 
				   Expression::type(self.rightHandSide)
\end{lstlisting}

Enfin, ce dernier contrat OCL (Listing \ref{lst:valid_void}) spécifie que le type de retour d'une procédure doit toujours être \textsf{void}.

\begin{lstlisting}[caption=Contrat OCL indiquant que le type de retour d'une procédure est toujours void,captionpos=b,label={lst:valid_void},language=OCL]
context Declaration::estValide()
	post:
		if self.oclIsTypeOf(Procedure) then
			return self.retour.oclIsTypeOf(Void)
\end{lstlisting}
