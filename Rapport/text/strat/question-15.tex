%% question-15.tex
%%

%% ==============================
\subsection{\textsc{Ocl} - Contrats sur l'opération \emph{prec\_mouv() : Déplacement}}
\label{sec:question15}
%% ==============================

Le premier contract OCL (Listing \ref{lst:action_move}) va permet de vérifier que lors d'un déplacement, si on tente d'accéder à une case ou se trouve un obstacle, le déplacement n'est pas effectué.

 \begin{lstlisting}[caption=On empêche le déplacement si la case est occupée par un obstacle,captionpos=b,label={lst:action_move},language=OCL]
context ProcedureAction::right()
	def: player = NiveauEnCours::elements.select(e -> e.name == "Cody"
					and e.oclIsTypeOf(Player))
	post rightObstacle:
		if 	NiveauEnCours::getElementAtPosition(player.Position.X + 1, 
			player.Position.Y).oclIsTypeOf(Obstacle) then
			player.Position.X == player.Position.X@pre and
			player.Position.Y == player.Position.Y@pre
			
context ProcedureAction::left()
	def: player = NiveauEnCours::elements.select(e -> e.name == "Cody"
					and e.oclIsTypeOf(Player))
	post leftObstacle:
		if 	NiveauEnCours::getElementAtPosition(player.Position.X - 1, 
			player.Position.Y).oclIsTypeOf(Obstacle) then
			player.Position.X == player.Position.X@pre and
			player.Position.Y == player.Position.Y@pre
			
context ProcedureAction::up()
	def: player = NiveauEnCours::elements.select(e -> e.name == "Cody"
					and e.oclIsTypeOf(Player))
	post upObstacle:
		if 	NiveauEnCours::getElementAtPosition(player.Position.X , 
			player.Position.Y - 1).oclIsTypeOf(Obstacle) then
			player.Position.X == player.Position.X@pre and
			pPlayer.Position.Y == player.Position.Y@pre
			
context ProcedureAction::down()
	def: player = NiveauEnCours::elements.select(e -> e.name == "Cody"
					and e.oclIsTypeOf(Player))
	post downObstacle:
		if 	NiveauEnCours::getElementAtPosition(player.Position.X, 
			player.Position.Y + 1).oclIsTypeOf(Obstacle) then
			player.Position.X == player.Position.X@pre and
			player.Position.Y == player.Position.Y@pre
\end{lstlisting}

Le contrat OCL (Listing \ref{lst:action_tunnel}) suivant indique que si le joueur accède à une case ou ce trouve un tunnel, il ressort dans la case suivant le dernier mouvement à partir de l'autre tunnel.

\begin{lstlisting}[caption=Contract OCL pour le passage dans un tunnel,captionpos=b,label={lst:action_tunnel},language=OCL]
context ProcedureAction::right()
	def: tunnel : Position
	def: player = NiveauEnCours::elements.select(e -> e.name == "Cody"
					and e.oclIsTypeOf(Player))
	post:
		tunnel = NiveauEnCours::getElementAtPosition(
		player.PositionX + 1, Player.PositionY)
			
		if tunnel.oclIsTypeOf(Tunnel) then
			player.Position.X == tunnel.Sortie.X + 1 
			and player.Position.Y == tunnel.Sortie.Y

context ProcedureAction::left()
	def: tunnel : Position
	def: player = NiveauEnCours::elements.select(e -> e.name == "Cody"
					and e.oclIsTypeOf(Player))
	post:
		tunnel = NiveauEnCours::getElementAtPosition(
		player.PositionX - 1, Player.PositionY)
			
		if tunnel.oclIsTypeOf(Tunnel) then
			player.Position.X == tunnel.Sortie.X - 1
			and player.Position.Y == tunnel.Sortie.Y
			
context ProcedureAction::up()
	def: tunnel : Position
	def: player = NiveauEnCours::elements.select(e -> e.name == "Cody"
					and e.oclIsTypeOf(Player))
	post:
		tunnel = NiveauEnCours::getElementAtPosition(
		player.PositionX, Player.PositionY - 1)
			
		if tunnel.oclIsTypeOf(Tunnel) then
			player.Position.X == tunnel.Sortie.X and
			player.Position.Y == tunnel.Sortie.Y - 1

context ProcedureAction::down()
	def: tunnel : Position
	def: player = NiveauEnCours::elements.select(e -> e.name == "Cody"
					and e.oclIsTypeOf(Player))
	post:
		tunnel = NiveauEnCours::getElementAtPosition(
		player.PositionX, Player.PositionY + 1)
			
		if tunnel.oclIsTypeOf(Tunnel) then
			player.Position.X == tunnel.Sortie.X and
			player.Position.Y == tunnel.Sortie.Y + 1
\end{lstlisting}

Enfin, le dernier contract OCL (Listing \ref{lst:action_jump}) permet d'indiquer que si le joueur saute, il atterit deux cases plus loin dans la même direction. Par contre, s'il y a un obstacle deux cases plus loin, le joueur ne bouge pas.

\begin{lstlisting}[caption=Contract OCL psur le saut,captionpos=b,label={lst:action_jump},language=OCL]
context ProcedureAction::jump()
	def: newPosition : Position
	def: player = Niveau::elements.select(e -> e.name == "Cody" and
					e.oclIsTypeOf(Player))
	post:
		if prec_mouv() == right then
			newPosition = NiveauEnCours::getElementAtPosition(
			player.PositionX + 2, player.PositionY)
			
			if newPosition.oclIsTypeOf(Surface) then
				player.Position.X == player.Position.X@pre + 2 
				and player.Position.Y == player.Position.Y@pre
			else
				player.Position.X == player.Position.X@pre 
				and player.Position.Y == player.Position.Y@pre
		if prec_mouv() == left then
			newPosition = NiveauEnCours::getElementAtPosition(
			player.PositionX - 2, player.PositionY)
			
			if newPosition.oclIsTypeOf(Surface) then
				player.Position.X == player.Position.X@pre - 2 
				and player.Position.Y == player.Position.Y@pre
			else		
				player.Position.X == player.Position.X@pre 
				and player.Position.Y == player.Position.Y@pre
		if prec_mouv() == up then
			newPosition = NiveauEnCours::getElementAtPosition(
			player.PositionX, player.PositionY - 2)
			
			if newPosition.oclIsTypeOf(Surface) then
				player.Position.X == player.Position.X@pre
				and player.Position.Y == player.Position.Y@pre
			else
				player.Position.X == player.Position.X@pre 
				and player.Position.Y == player.Position.Y@pre - 2
		if prec_mouv() == down then
			newPosition = NiveauEnCours::getElementAtPosition(
			player.PositionX, player.PositionY + 2)
			
			if newPosition.oclIsTypeOf(Surface) then
				player.Position.X == player.Position.X@pre 
				and player.Position.Y == player.Position.Y@pre + 2
			else
				player.Position.X == player.Position.X@pre 
				and player.Position.Y == player.Position.Y@pre
\end{lstlisting}