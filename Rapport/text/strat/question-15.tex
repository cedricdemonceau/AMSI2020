%% question-15.tex
%%

%% ==============================
\subsection{\textsc{Ocl} - Contrainte sur les déclarations}
\label{sec:question15}
%% ==============================

La première contrainte sur l'unicité des littéraux est la suivante :

\begin{lstlisting}[caption=unicité des litéraux,captionpos=b,label={lst:lit_unique},language=OCL]
context enumeration inv literauxUnique :
	self.Literal.allInstances -> forall( l1,l2 | 
		l1.nom <> l2.nom implies l1 <> l2)
\end{lstlisting}

La contrainte sur la liste des champs est la suivante :

\begin{lstlisting}[caption=champ non vide,captionpos=b,label={lst:champ},language=OCL]
context Enregistrement inv champNonVide :
	self.Champ.allInstances -> size() > 0
\end{lstlisting}

La contrainte sur l'unicité du nom des champs est la suivante :

\begin{lstlisting}[caption=Nom unique des champs,captionpos=b,label={lst:champ_unique},language=OCL]
context Enregistrement inv nomChamp :
	self.Champ.allInstances -> forall( c1,c2 | 
	c1.nom <> c2.nom implies c1 <> c2)
\end{lstlisting}

La contrainte sur la dimension d'un tableau est la suivante :

\begin{lstlisting}[caption=dimension d'un tableau,captionpos=b,label={lst:dim_tableau},language=OCL]
context Tableau inv simension :
	self.longueur > 0
\end{lstlisting}

La contrainte sur la dimension positive des tableaux est la suivante :

\begin{lstlisting}[caption=Dimension positive,captionpos=b,label={lst:dim_positive},language=OCL]
context Tableau inv dimPositive :
	self.allInstances -> forall(t | t.possede.longueur > 0)
\end{lstlisting}