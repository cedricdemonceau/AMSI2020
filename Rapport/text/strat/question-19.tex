%% question-19.tex
%%

%% ==============================
\subsection{Opération Instruction :: estValide : Boolean}
\label{sec:question19}
%% ==============================

Le contrat d'une telle opération est le suivant :

- L'instruction "skip" est toujours valide :

\begin{lstlisting}[caption=Contrat \textsc{Ocl} sur l'opération estValide,captionpos=b,label={lst:valide},language=OCL]
context Instruction :: estValide() : Boolean
    post :
        if Instruction.oclIsTypeOf(Skip)
        then
            true
        endif
\end{lstlisting}

- La garde d'une \emph{Conditionelle} ou d'une \emph{Iteration} est booleéne :

\begin{lstlisting}[caption=Contrat \textsc{Ocl} sur la garde,captionpos=b,label={lst:garde},language=OCL]
context Instruction :: estValide() : Boolean
    post :
        if Instruction.oclIsTypeOf(Conditionelle) or
            Instruction.oclIsTypeOf(Iteration)
        then
            Instruction.garde.oclIsTypeOf(Boolean)
        endif
\end{lstlisting}

- La partie gauche et droite d'une \emph{Affectation} sont du même type :

\begin{lstlisting}[caption=Contrat \textsc{Ocl} les parties d'une affectation,captionpos=b,label={lst:parties},language=OCL]
context Instruction :: estValide() : Boolean
    post :
        if Instruction.oclIsTypeOf(Affectation)
        then
            Instruction.leftHandSide.oclIsTypeOf(rightHandSide)
        endif
\end{lstlisting}