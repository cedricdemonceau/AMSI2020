%% question-14.tex
%%

%% ==============================
\subsection{\textsc{Ocl} - Contraintes d'unicité}
\label{sec:question14}
%% ==============================

La première contrainte d'unicité sur le nom unique des modules est spécifiée comme suit :

\begin{lstlisting}[caption=Nom unique au sein d'une stratégie,captionpos=b,label={lst:nom_unique},language=OCL]
context Strategie inv nomUnique :
	if self.Declaration.allInstances -> oclIsTypeOf(Module) and
		self.Declaration.forall( m1, m2 | m1.nom <> m2.nom)
	then
		true
	else
		false
	endif
\end{lstlisting}

La contrainte sur le nom des variables globales est la suivante :

\begin{lstlisting}[caption=Nom unique d'une variable globale,captionpos=b,label={lst:unique_globale},language=OCL]
context Strategie inv nomVarGlobal :
	if self.Declaration.allInstances ->oclIsTypeOf(Variable) and
		self.Declaration.forall( v1,v2 | v1.nom <> v2.nom)
	then
		true
	else
		false
	endif
\end{lstlisting}

La contrainte sur les variables locales est la suivante :

\begin{lstlisting}[caption=Nom unique des variables locales,captionpos=b,label={lst:nom_locale},language=OCL]
context Module inv uniqueLocale :
	self.Variable.allInstances -> forall(v1,v2 | 
	v1.nom <> v2.nom implies v1 <> v2)
\end{lstlisting}

La contrainte sur les noms des paramètres d'un module est la suivante :

\begin{lstlisting}[caption=Nom unique des paramètres,captionpos=b,label={lst:unique_param},language=OCL]
context Module inv uniqueParam :
	self.Parametre.allInstances -> forall(p1,p2 | 
	p1.nom <> p2.nom implies p1 <> p2)
\end{lstlisting}

La contrainte sur le nom des types soit globalement unique est la suivante :

\begin{lstlisting}[caption=Nom unique des types,captionpos=b,label={lst:type_unique},language=OCL]
context Type inv difNom :
	self.allInstances -> forall( e,t | 
	e.oclIsTypeOf(Enumeration) and t.oclIsTypeOf(Tableau) 
	and e.nom <> t.nom)
\end{lstlisting}