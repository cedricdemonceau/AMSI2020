%% question-16.tex
%%

%% ==============================
\subsection{\textsc{Ocl} - Contrats sur l'opération \emph{Expression :: type() : Type}
\label{sec:question16}
%% ==============================

Ce premier contrat porte sur le fait que le type du litéral retourné corresponde au litéral :

\begin{lstlisting}[caption=Contrat \textsc{Ocl} sur le type des litéraux,captionpos=b,label={lst:type_literal},language=OCL]
context Expression :: type() : Type
	post:
		if Type.oclIsTypeOf(TypePrimitif) then
			return Type.oclType()
\end{lstlisting}

Le contrat sur l'opérateur unaire est la suivante :

\begin{lstlisting}[caption=Contrat \textsc{Ocl} sur le type unaire,captionpos=b,label={lst:type_unaire},language=OCL]
context Expression :: type() : Type
	post:
		if Type.oclIsTypeOf(Unaire) then
			if Type.operateur.oclIsTypeOf(
				Type.sous-expression.oclType()
			) then
				return Type.operateur.oclType()
\end{lstlisting}

Le contrat correspondant à l'opérateur binaire est le suivant :

\begin{lstlisting}[caption=Contrat \textsc{Ocl} sur le type binaire,captionpos=b,label={lst:type_binaire},language=OCL]
context Expression :: type() : Type
	post:
		if Type.oclIsTypeOf(Binaire) then
			if Type.operateur.oclIsTypeOf(
				Type.leftHandSide.oclType()
			) and Type.operateur.oclIsTypeOf(
				Type.rightHandSide.oclType()
			) then
				return Type.operateur.oclType()
\end{lstlisting}

Le contrat sur le type de la sous-expression dans une parenthèse est le suivant :

\begin{lstlisting}[caption=Contrat \textsc{Ocl} sur le type paranthésée,captionpos=b,label={lst:type_paranthese},language=OCL]
context Expression :: type() : Type
	post:
		if Type.oclIsTypeOf(Parenthese) then
			return Type.sous-expression.oclType()
\end{lstlisting}

Le contrat sur le type de la déclaration est le suivant :

Le contrat sur le type de renvoi d'une énumération de litéraux est le suivant :

\begin{lstlisting}[caption=Contrat \textsc{Ocl} sur le type de renvoi d'une énum de litéraux,captionpos=b,label={lst:type_énum},language=OCL]
context Expression :: type() : Type
	post :
		if Type.oclIsTypeOf(Litteral)
		then
			return
				Type.est.oslType()
\end{lstlisting}

Le contrat sur le type de retour d'un champ est le suivant :

\begin{lstlisting}[caption=Contrat \textsc{Ocl} sur le type de retour d'un champ,captionpos=b,label={lst:type_champ},language=OCL]
context Expression :: type() : Type
	post :
		if Type.oclIsTypeOf(Champ)
		then
			return
			Type.TypePrimitif.oclType()
\end{lstlisting}

Le contrat sur le type du contenu d'un tableau est le suivant :

\begin{lstlisting}[caption=Contrat \textsc{Ocl} sur le type de retour d'un tableau,captionpos=b,label={lst:type_tableau},language=OCL]
context Expression :: type() : Type
	post :
		if Type.oclIsTypeOf(Tableau)
		then
		 return
		 	Type.est.oclType()
\end{lstlisting}