%% question-4.tex
%%

%% ==============================
\subsection{\textsc{Ocl} - Contraintes OCL du diagramme de classe}
\label{sec:question-4}
%% ==============================


Le fait que le potentiel de vie d'un joueur est toujours positif est caractérisé par la contrainte suivante:

\begin{lstlisting}[caption=Contrainte sur lpotentiel de vie,captionpos=b,label={lst:vie},language=OCL]
context Personnage inv vie :
	self.potentielDeVie > 0
\end{lstlisting}

La contrainte exprimant le fait que les ratios d'attaque s'exprime comme suit:

\begin{lstlisting}[caption=Contrainte sur les ratios,captionpos=b,label={lst:ratios},language=OCL]
context Personnage inv ratios :
	self.ratioAttaque > 0 and self.ratioAttaque < 1
		and
			if self.oclIsTypeOf(Avatar)
				then self.ratioDefense > 0 and self.ratioDefense < 1
			endif
\end{lstlisting}

La contrainte sur l'imparité des rencontres est caractérisée comme suit:

\begin{lstlisting}[caption=Contrainte sur l'imparité des rencontres,captionpos=b,label={lst:impair},language=OCL]
context Match inv nbRencontre :
	self.rencontre.size() % 2 = 1
\end{lstlisting}

La contrainte exprimant que le numéro identifiant la rencontre correspond à son ordre de jeu est la suivante:

\begin{lstlisting}[caption=Contrainte sur l'ordre des rencontres,captionpos=b,label={lst:ordreRencontres},language=OCL]
context Match inv ordoné :
	self.rencontre.allInstances -> asOrderedSet()
\end{lstlisting}

La contrainte sur le les 3 types d'items contenu par l'inventaire est représentée dans le diagramme de classe par l'héritage d'\emph{Objet} \ref{fig:Jeu}.

La contrainte qu'un monde ne possède qu'un \emph{Graal} est aussi représentée dans le diagramme de classe par l'association possède entre \emph{Monde} et \emph{Graal} \ref{fig:Jeu}.

La contrainte du fait que les cases ne peuvent exéder la longueur d'un monde est la suivante:

\begin{lstlisting}
context Monde inv posCases :
	self.case.allInstances -> forall ( c |
		c.position.x >= 0 and c.position.x < self.tailleX and
		c.position.y >= 0 and c.position.y < self.tailleY)
\end{lstlisting}

Le fait que le joueur se trouve dans la case correspondant à sa position absolue est caractérisé par la contrainte suivante :

\begin{lstlisting}
context Avatar inv posAbsolue :
	self.rencontre.monde.case.allInstances -> forall ( c1,c2 |
		self.position.x = c1.position.x and self.position.y = c1.position.y implies
		self.position.x <> c2.position.x and self.position.y <> c2.position.y)
\end{lstlisting}

