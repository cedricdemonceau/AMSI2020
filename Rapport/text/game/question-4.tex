%% question-4.tex
%%

%% ==============================
\subsection{\textsc{Ocl} - Contraintes OCL du diagramme de classe}
\label{sec:question-4}
%% ==============================


Le fait que le potentiel de vie d'un joueur est toujours positif est caractérisé par la contrainte suivante:

\begin{lstlisting}[caption=Contrainte sur le potentiel de vie,captionpos=b,label={lst:vie},language=OCL]
context Personnage inv vie :
	self.potentielDeVie > 0
\end{lstlisting}

La contrainte exprimant le fait que les ratios d'attaque et de défense sont des ratios s'exprime comme suit:

\begin{lstlisting}[caption=Contrainte sur les ratios,captionpos=b,label={lst:ratios},language=OCL]
context Personnage inv ratios :
	self.ratioAttaque > 0 and self.ratioAttaque < 1
		and
			if self.oclIsTypeOf(Avatar)
				then self.ratioDefense > 0 and self.ratioDefense < 1
			endif
\end{lstlisting}

La contrainte sur la non-parité des rencontres est caractérisée comme suit:

\begin{lstlisting}[caption=Contrainte sur la non-parité des rencontres,captionpos=b,label={lst:impair},language=OCL]
context Match inv nbRencontre :
	self.rencontre.size() % 2 = 1
\end{lstlisting}

La contrainte exprimant que le numéro identifiant la rencontre correspond à son ordre de jeu est la suivante:

\begin{lstlisting}[caption=Contrainte sur l'ordre des rencontres,captionpos=b,label={lst:ordreRencontres},language=OCL]
context Match inv ordone :
	self.rencontre.allInstances() -> asOrderedSet()
\end{lstlisting}

La contrainte sur le les 3 types d'items contenu par l'inventaire est représentée dans le diagramme de classe par l'héritage d'\emph{Objet} (figure \ref{fig:Jeu}).

\vspace*{2cm}

La contrainte qu'un monde ne possède qu'un \emph{Graal} est aussi représentée dans le diagramme de classe par l'association possède entre \emph{Monde} et \emph{Graal} (figure\ref{fig:Jeu}).

\vspace*{2cm}

La contrainte du fait que les cases ne peuvent excéder la longueur d'un monde est la suivante:

\begin{lstlisting}[caption=Contrainte sur la position des cases,captionpos=b,label={lst:casePos},language=OCL]
context Monde inv posCases :
	self.case.allInstances() -> forall ( c |
		c.position.x >= 0 and c.position.x < self.tailleX and
		c.position.y >= 0 and c.position.y < self.tailleY)
\end{lstlisting}

La contrainte  sur la disposition des cases est la suivante :


Le fait que le joueur se trouve dans la case correspondant à sa position absolue est caractérisé par la contrainte suivante :

\begin{lstlisting}[caption=Contrainte sur la position du joueur,captionpos=b,label={lst:posJoueur},language=OCL]
context Avatar inv posAbsolue :
	self.rencontre.monde.case.allInstances() -> forall ( c1,c2 |
		self.position.x = c1.position.x and 
		self.position.y = c1.position.y implies
		self.position.x <> c2.position.x and 
		self.position.y <> c2.position.y)
\end{lstlisting}

Le fait que la bordure d'un plateau contienne toujours des éléments infranchissables est caractérisé par la contrainte suivante :

\begin{lstlisting}[caption=Contrainte sur la bordure du plateau,captionpos=b,label={lst:bordurePlat},language=OCL]
context Monde inv Infranchissable :
	self.case.allInstances() -> forall(c |
		(c.position.x = 0 and c.position.y <= 0 and 
		c.position.y > self.tailleY or
		c.position.x = self.tailleX - 1 and 
		c.position.y <= 0 and c.position.y > self.tailleY or
		c.position.y = 0 and c.position.x <= 0 and 
		c.position.x > self.tailleX or
		c.position.y = self.tailleY - 1 and c.position.x <= 0 and
		 c.position.x > self.tailleX) implies
			self.case.oclIsTypeOf(Infranchissable)
\end{lstlisting}

La contrainte sur la visibilité est la suivante :

Le fait qu'un personnage ne puisse se trouver sur une case portant un obstacle infranchissable est caractérisé par la contrainte suivante :

\begin{lstlisting}[caption=Contrainte sur la position du joueur sur case infranchissable,captionpos=b,label={lst:caseInfranchissable},language=OCL]
context Monde inv posJoueurCase :
	if self.case.allInstances() -> oclIsTypeOf(Infranchissable) and
		self.case.position.x = self.rencontre.avatar.position.x and
		self.case.position.y = self.rencontre.avatar.position.y
	then
		false
	else
		true
	endif
\end{lstlisting}

Le fait que 2 personnages ( zombie compris) ne puissent se trouver sur la même case est caractérisé par la contrainte suivante :

\begin{lstlisting}[caption=Contrainte sur la position de deux personnages,captionpos=b,label={lst:posPersonnages},language=OCL]
context Personnage inv memeCase :
	self.allInstances() -> forall ( p1,p2 |
		p1.position.x = p2.position.x and 
		p1.position.y = p2.position.y implies
		p1 = p2)
\end{lstlisting}