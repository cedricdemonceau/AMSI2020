%% question-3.tex
%%

%% ==============================
\subsection{\textsc{Ocl} - Contraintes d'unicité}
\label{sec:question3}
%% ==============================

Le fait que les matchs sont identifiés de manière unique se caractérise par cette contrainte Ocl:

\begin{lstlisting}[caption=Contrainte sur l'unicité d'un match,captionpos=b,label={lst:match},language=OCL]
context Jeu inv matchUnique :
    self.Match.allInstances -> forall( m1,m2 |
        m1.id <> m2.id implies m1 <> m2)
\end{lstlisting}

La contrainte que des joueurs ont des pseudos différents au sein du jeu se caractérise par cette contrainte :

\begin{lstlisting}[caption=Contrainte sur les pseudos,captionpos=b,label={lst:pseudos},language=OCL]
context Jeu inv pseudoJoueur :
    self.Joueurs.allInstances -> forall( j1,j2 |
        j1.pseudo <> j2.pseudo implies j1 <> j2)
\end{lstlisting}

La contrainte Ocl disant que les personnages/avatars d'un joueur sont nommés différement est exprimée comme suit:

\begin{lstlisting}[caption=Contrainte sur le nom,captionpos=b,label={lst:nomAvatar},language=OCL]
context Joueurs inv nomAvatars :
    self.Avatar.allInstances -> forall( a1,a2 |
        a1.nom <> a2.nom and a1.aspect <> a2.aspect
            implies a1 <> a2)
\end{lstlisting}

La contrainte Ocl obligeant les rencontres à avoir un numéro d'ordre unique est la suivante :

\begin{lstlisting}[caption=Contrainte sur le numéro d'ordre unique,captionpos=b,label={lst:numUnique},language=OCL]
context Match inv ordre :
    self.rencontres.allInstances -> forall( r1,r2 |
        r1.numeroOrdre <> r2.numeroOrdre implies r1 <> r2)
\end{lstlisting}